%%%%%%%%%%%%%%%%%%%%%%%%%%%%%%%%%%%%%%%%%%%%%%%%%%%%%%%%%%%%%%%%%%%%%
% LaTeX Template: Praxisprojekt WS 2017
%
% Date: November 2017
%
%%%%%%%%%%%%%%%%%%%%%%%%%%%%%%%%%%%%%%%%%%%%%%%%%%%%%%%%%%%%%%%%%%%%%%

\documentclass[12pt]{article}
\usepackage[a4paper]{geometry}
\usepackage{framed}
\usepackage[myheadings]{fullpage}
\usepackage{fancyhdr}
\usepackage{lastpage}
\usepackage{graphicx, wrapfig, subcaption, setspace, booktabs}
\usepackage[T1]{fontenc}
\usepackage[font=small, labelfont=bf]{caption}
\usepackage[protrusion=true, expansion=true]{microtype}
\usepackage[german]{babel}
\usepackage{sectsty}
\usepackage{url, lipsum}
\usepackage[parfill]{parskip}
\usepackage{listings}


\usepackage{enumitem,amssymb}
\newlist{todolist}{itemize}{2}
\setlist[todolist]{label=$\square$}
\usepackage{pifont}
\newcommand{\cmark}{\ding{51}}%
\newcommand{\xmark}{\ding{55}}%
\newcommand{\done}{\rlap{$\square$}{\raisebox{2pt}{\large\hspace{1pt}\cmark}}%
\hspace{-2.5pt}}
\newcommand{\wontfix}{\rlap{$\square$}{\large\hspace{1pt}\xmark}}

\usepackage{dirtree}
\DTsetlength{ 0.2em}{ 1em}{ 0.2em}{ 0.4pt}{ 0.1pt}

%-------------------------------------------------------------------------------
% Commands
%-------------------------------------------------------------------------------
\newcommand{\HRule}[1]{\rule{\linewidth}{#1}}
\input{env}
%-------------------------------------------------------------------------------
% HEADER & FOOTER
%-------------------------------------------------------------------------------
\pagestyle{fancy}
\fancyhf{}
\setlength\headheight{15pt}
\fancyhead[L]{\newCommandName}
\fancyhead[R]{\newCommandUniversity}
\fancyfoot[R]{Seite \thepage\ von \pageref{LastPage}}

%-------------------------------------------------------------------------------
% TITLE PAGE
%-------------------------------------------------------------------------------
\begin{document}


\title{ \normalsize
		\HRule{0.5pt} \\
		\LARGE \textbf{\uppercase{\newCommandDiscipline}} \\
		\smallbreak
		\small\textbf{{\newCommandTerm}}\\
		\HRule{2pt} \\ [0.5cm]
		\normalsize \today \vspace*{10\baselineskip}}

\date{}

\author{
		\newCommandName \\
		\newCommandMatriculationNumber \\
		\newCommandUniversity \\
		\newCommandFaculty
}

\pagenumbering{gobble}

\maketitle

\newpage

\pagenumbering{arabic}


%-------------------------------------------------------------------------------
% Section title formatting
\sectionfont{\scshape}
%-------------------------------------------------------------------------------

%-------------------------------------------------------------------------------
% BODY
%-------------------------------------------------------------------------------

\section{Statusbericht}
\subsection{Was habe ich bisher im Projekt erreicht?}
\subsubsection{Checklist}
\begin{itemize}
	\item Rest Resource Job implementiert
	\item Rest Resource Template implementiert
	\item Jeder Job hat beliebige viele Templates welche spaeter die Variantensimulationen repraesentieren
	\item Als Persistence-Layer wurde ein Repository zur MongoDB implementiert
	\item Router mit diversen Middleware-Handlern implementiert (Logger, Auth)
	\item Ich habe mich ausfuehrlich in Best-Practices zur Architektur von Micro-Services eingearbeitet
	\item Refactoring des API-Designs
	\item TDD und BDD angefangen umzusetzen
\end{itemize}
\smallbreak

\subsubsection{Dir Tree}

\renewcommand*\DTstylecomment{\rmfamily\color{red}\textsc}
\dirtree{%
.1 / .
.2 cmd .
	.3 server .
		.4 hpc-rest-api.go .
.2 src .
	.3 connection .
		.4 config.go .
	.3 handler .
		.4 authhandler.go .
		.4 jobhandler.go .
		.4 templatehandler.go .
		.4 logginghandler.go .
	.3 repository .
		.4 jobrepository.go .
		.4 templaterepository.go .
	.3 router .
		.4 router.go .
}
\newpage
\subsubsection{Routing Endpoints}
\smallbreak
\begin{table}[h]
\begin{tabular}{||l|r|l||}
	\hline
 \textbf{URI-Endpoint}  & \textbf{HTTP-Method} & \textbf{Description} \\ \hline\hline
 \texttt{/jobs/} & GET & Get all Jobs \\ \hline
 \texttt{/jobs/} & POST & Create new Job \\ \hline
 \texttt{/jobs/} & PUT & Update Job \\ \hline
 \texttt{/jobs/} & DELETE & Delete Job \\ \hline
 \texttt{/jobs/\{jobID\}} & GET & Get specific Job \\ \hline \hline
 \texttt{/jobs/\{jobID\}/templates/} & GET & Get all Templates \\ \hline
 \texttt{/jobs/\{jobID\}/templates/} & POST & Create new Template \\ \hline
 \texttt{/jobs/\{jobID\}/templates/} & PUT & Update Template \\ \hline
 \texttt{/jobs/\{jobID\}/templates/} & DELETE & Delete Template \\ \hline
 \texttt{/jobs/\{jobID\}/templates/\{templateID\}} & GET & Get specific Template \\ \hline
\end{tabular}
\caption{Routing Endpoints (Job \& Template)}
\label{tab:meinetabelle}
\end{table}




\subsection{Was habe ich bis zum n"achsten Statusbericht vor?}


\begin{itemize}
\item NOCH OFFEN: Kompletter Durchlauf der Api-Architectur
\begin{todolist}
  \item[\done] Job CRUD
  \item[\done] Template CRUD
  \item FileUpload und Anbindung Minio
  \item Job-Submit Endpoint
	\item Solver Gateway - Nomad Scheduling of Jobs
  \item Persist Results in Minio Objectstorage
\end{todolist}
\item Dokumentation der Api
\item TDD implementieren und 100\% Code-Coverage erreichen
\end{itemize}

Dokumentation der Issues werden direkt im Repository gef"uhrt.
\subsection{Gibt es Probleme bei der Durchf"uhrung?}
\begin{itemize}
\item Es gab in den letzten Wochen ein Problem ein richtiges Zeitmanagement zu finden. Also wieviel Zeit wende ich fuer das Erlernen neuer Sachen auf und wieviel fuer die Anwendung/Umsetzung.
\item Beim TDD/BDD habe ich noch Probleme ein geignetes MockingFramework zu finden.
\end{itemize}


\section{Ressourcen}
\textbf{Git-Repository als Versionskontrolle:}\\
\url{https://github.com/MatthiasHertel/ws17_pp}

\textbf{Webseite zur Dokumentation:}\\
\url{https://www.ws17-pp.mhertel.de}


\end{document}
